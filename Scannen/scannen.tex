\documentclass{slides}
\usepackage{german}
\usepackage{epsfig}
\usepackage{amssymb}

\pagestyle{empty}
\setlength{\textwidth}{17cm}
\setlength{\textheight}{24cm}
\setlength{\topmargin}{0cm}
\setlength{\headheight}{0cm}
\setlength{\headsep}{0cm}
\setlength{\topskip}{0cm}
\setlength{\oddsidemargin}{0cm}
\setlength{\evensidemargin}{0cm}

\newcommand{\Ll}{{\cal L}}
\newcommand{\Rl}{{\cal R}}
\newcommand{\NS}{{\cal N\!S}}
\newcommand{\cl}[1]{{\cal #1}}

\newcommand{\bruch}[2]{\frac{\displaystyle\;#1\;}{\displaystyle\;#2\;}}
\newcommand{\Oh}{\mathcal{O}}


\newcounter{mypage}

\begin{document}

\begin{center}
Pattern Matching
\end{center}


\rule{17cm}{1mm}

\footnotesize
\textbf{Definitionen}:
\begin{enumerate}
\item \emph{Alphabet}: endliche Menge von Zeichen, die \\
      \emph{Buchstaben} genannt werden.
\item \emph{W"orter}:  Sei $\Sigma$ Alphabet.  Ein $\Sigma$--\emph{Wort} ist endliche Folge von
      Buchstaben aus $\Sigma$: \\[0.3cm]
      \hspace*{1.3cm} $w = b_1b_2 \cdots b_n$ \quad mit $b_i \in \Sigma$ \\[0.3cm]
      ist $\Sigma$--Wort der L"ange $n$

      $\Sigma^*$: Menge der $\Sigma$--W"orter
\item Das \emph{leere Wort} bezeichnen wir mit $\varepsilon$.
\item \emph{Konkatenation} von W"ortern: 

      Seien $v=b_1b_2 \cdots b_m \in \Sigma^*$ und $w=c_1c_2 \cdots c_n \in \Sigma^*$.

      Wir definieren die Konkatenation $vw$ von $v$ und $w$ als \\[0.3cm]
      \hspace*{1.3cm} $vw := b_1b_2 \cdots b_mc_1c_2 \cdots c_n$
\item \emph{Sprache}:

      Eine beliebige Teilmenge ${\cal L} \subseteq \Sigma^*$ hei{\ss}t $\Sigma$--\emph{Sprache}.
\end{enumerate}
\textbf{Beispiel}: Sei $\Sigma = \{a,b\}$.  Dann gilt 
\begin{enumerate}
\item $\Sigma^* = \{ \varepsilon, a, b, aa, ab, ba, bb, aaa, \cdots \}$ 
\item ${\cal L}_{Q} := \{ w \in \Sigma^* \;|\; \exists v \in \Sigma^*: w = vv \}$ 

      ist eine $\Sigma$--Sprache
\end{enumerate}
\hspace*{1.3cm} 


\vspace*{0.2cm}

\scriptsize

\vspace*{\fill}
\tiny \addtocounter{mypage}{1}
\rule{17cm}{1mm}
Pattern Matching  \hspace*{\fill} Seite \arabic{mypage}


\include{scannen-slides}
\end{document}

%%% Local Variables: 
%%% mode: latex
%%% TeX-master: t
%%% End: 
